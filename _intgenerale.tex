\chapter*{Introduction générale}
\markboth{Introduction générale}{}
\addcontentsline{toc}{chapter}{Introduction générale}
\thispagestyle{plain}

\section*{Contexte générale}
\addcontentsline{toc}{section}{Contexte générale}

Les parkings ont vu le jour à la fin du 19e siècle avec l'introduction des premières automobiles dans les villes du monde. Au départ, peu d'emplacements de stationnement spécifiques étaient nécessaires en raison du faible nombre de voitures, et les gens se garaient sur les bords de route ou dans des garages privés. Cependant, à mesure que le nombre de voitures augmentait et que les villes s'étendaient, le stationnement est devenu un enjeu crucial.

\section*{Problématique et motivation}
\addcontentsline{toc}{section}{Problématique et motivation}

La problématique qui se pose est la suivante : Comment optimiser la gestion des parkings pour répondre à la demande croissante de stationnement tout en minimisant les problèmes liés à la congestion, à la disponibilité limitée d'espace et à l'impact environnemental ? Cette question est motivée par la nécessité de relever ces défis complexes dans un contexte urbain en constante évolution. Les parkings jouent un rôle crucial dans la mobilité urbaine, l'accès aux services et le confort des citoyens, ce qui rend essentielle l'amélioration de leur gestion. Par conséquent, il est impératif de développer des solutions innovantes pour améliorer la gestion des parkings, en tirant parti des avancées technologiques récentes telles que l'intelligence artificielle.

\section*{Contribution}
\addcontentsline{toc}{section}{Contribution}

La contribution essentielle de ce travail réside dans le développement d'un système de gestion de stationnement en temps réel, basé sur le modèle YOLOv8. Ce système automatisé est capable de reconnaître les véhicules en mouvement ou à l'arrêt, de détecter les plaques d'immatriculation, et de calculer les places de stationnement disponibles et occupées, le tout sans nécessiter d'intervention humaine.

En outre, nous avons constitué une base de données réelle comprenant des images de véhicules et de plaques d'immatriculation Algériennes.

\section*{Plan du mémoire}
\addcontentsline{toc}{section}{Plan du mémoire}
Ce manuscrit se structure autour de quatre chapitres fondamentaux, à savoir :

\paragraph{Chapitre 1 :}
Gestion du stationnement.\\
 Dans le premier chapitre, nous explorons le contexte général de la gestion des parkings.

\paragraph{Chapitre 2 :}
Apprentissage profond et détection d'objet: Concepts et algorithmes.\\
Dans le second chapitre, nous abordons particulièrement le concept de l'apprentissage profond et ses divers paradigmes, ainsi que les principaux algorithmes liés à la détection d'objets dans ce domaine.

\paragraph{Chapitre 3 :}
Gestion automatisée du stationnement: État de l'art.\\
Dans le troisième chapitre, nous évoquons les principales caractéristiques des plaques d'immatriculation des voitures algériennes. Nous passons également en revue les approches employées pour la détection des véhicules, la reconnaissance des plaques d'immatriculation, l'identification des caractères numériques sur les plaques, ainsi que le suivi en temps -réel des véhicules.

\paragraph{Chapitre 4 :}
Système intelligent de gestion du parking: Réalisation, résultats et analyse.\\
Le quatrième chapitre est consacré à la présentation de la solution proposée et des résultats obtenus par le système de gestion de parking intelligent à base du modèle YOLOv8.