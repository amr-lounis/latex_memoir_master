\begin{RLtext}
\par
\textbf{\Large{ملخص}}\\
\small

مع النمو الكبير في عدد السيارات حول العالم أصبح الطلب المتزايد على مواقف السيارات في المدن الكبيرة أمرًا ضروريًا نظرًا لندرة هذه المساحات ووجود أساليب الإدارة التقليدية  هذا الوضع يستدعي اعتماد تكنولوجيات جديدة لإدارة مواقف السيارات بطريقة ذكية  مما يضمن التنظيم الفعّال والاستفادة الأمثل من هذه الموارد  الهدف الرئيسي هو الاستفادة الكاملة من المساحة المخصصة  واستيعاب المزيد من السيارات  وضمان راحة السائقين، وتقليل ازدحام حركة المرور 
\par
الإدارة التقليدية لمواقف السيارات عادةً ما تتضمن تسجيل أوقات دخول وخروج السيارات مع لوحات تراخيصها مع الحفاظ على تتبع دقيق للمساحات المتاحة  يتم تكليف مهام هذه المسؤوليات عادة لموظف استقبال يستقبل 
السيارات عند وصولها، ويفحص لوحات التراخيص، ويسجل المعلومات اللازمة 


\par
في هذا السياق، يركز عملنا على تطوير نظام ذكي لإدارة مواقف السيارات إعتمادا على الذكاء الاصطناعي. يمكن لهذا النظام الآلي التعرف على السيارات في حالة الحركة أو الوقوف، واكتشاف وقراءة لوحات التراخيص، وحساب عدد  المساحات المتاحة والمحتلة، دون تدخل بشري. باستخدام النموذج \LR{YOLOv8}، اختبرنا هذا الحل باستخدام صور للسيارات الجزائرية ولوحات تراخيصها التي تم جمعها من مصادر مثل \LR{Ouedkniss}  و \LR{Facebook}.  ثم قمنا بتقييم أداء النظام من خلال مقارنته مع  النمودجين \LR{SSD} و \LR{Faster RCNN}. أظهرت النتائج المحصل عليها أنها واعدة.


\textbf{الكلمات المفتاحية \LR{:}} 
 نظام الوقت الحقيقي ، إدارة مواقف السيارات، الكشف عن المركبات ، التعرف على لوحة الترخيص ، قراءة لوحة الترخيص ،  تتبع المركبات, \LR{YOLOV8} , \LR{SSD} , \LR{FASTER R-CNN} .
\end{RLtext}

\vspace{0.5cm}

\par
\textbf{\Large{Abstract}}\\

With the significant growth in the number of vehicles worldwide, the increasing demand for parking in metropolises has become imperative due to the scarcity of these spaces and the presence of traditional management methods. This situation necessitates the adoption of new technologies for intelligent parking management, ensuring efficient organization and optimal utilization of these resources. The primary goal is to fully exploit the dedicated parking space, accommodate more vehicles, ensure driver comfort, and reduce traffic congestion.

Conventional parking management typically involves recording vehicle entry and exit times along with their license plates while maintaining precise tracking of available spaces. These responsibilities are usually assigned to an attendant who receives vehicles upon arrival, checks license plates, and records the necessary information.

In this context, our work focuses on developing an intelligent parking management system based on artificial intelligence. This automated system can recognize vehicles in motion or stationary, it can also detect and read license plates, and calculate the number of available and occupied parking spaces, all without human intervention. Using the YLOLOv8 model, we tested this solution using images of Algerian vehicles and license plates collected from sources such as Ouedkniss and Facebook. We subsequently assessed the system's performance by comparing it to Faster RCNN and SSD detectors. The results obtained proved to be promising.


\textbf{Keywords}: Real-time system, parking management, vehicle detection, license plate recognition, vehicle tracking and object detector.

\vspace{0.5cm}

\par
\textbf{\Large{Résumé}}\\

Avec la croissance significative du nombre de véhicules dans le monde, l'augmentation de la demande de stationnement dans les métropoles s'impose en raison de la rareté de ces espaces et des méthodes de gestion traditionnelles en place. Cette situation nécessite l'adoption de nouvelles technologies pour une gestion intelligente des parkings, assurant ainsi une organisation efficace et une utilisation optimale de ces ressources. L'objectif principal est d'exploiter pleinement l'espace dédié au stationnement, d'accueillir davantage de véhicules, de garantir le confort des conducteurs et de réduire la congestion routière.
\par

La gestion du stationnement conventionnelle implique généralement l'enregistrement des heures d'entrée et de sortie des véhicules ainsi que de leurs plaques d'immatriculation, tout en assurant un suivi précis des places disponibles. Ces responsabilités sont habituellement dévolues à un agent qui se charge de recevoir les véhicules à leur arrivée, de vérifier les plaques d'immatriculation et de saisir les informations nécessaires.

\par

Dans ce contexte, notre mémoire se concentre sur le développement d'un système de gestion de stationnement intelligent basé sur l'intelligence artificielle. Ce système automatisé est capable de reconnaître les véhicules en mouvement ou à l'arrêt, de détecter les plaques d'immatriculation et de calculer le nombre de places disponibles et occupées dans les parkings, le tout sans intervention humaine. En utilisant le modèle Yolov8, nous avons expérimenté cette solution en utilisant des images de véhicules et de plaques d'immatriculation d'origine Algérienne que nous avons collectées à partir de sources telles qu'Ouedkniss et Facebook. Nous avons ensuite évalué les performances de ce système en le comparant avec les détecteurs Faster RCNN et SSD. Les résultats obtenus se sont avérés prometteurs.

\par \textbf{Mots-clés} : Système temps-réel, gestion du stationnement, détection de véhicules, reconnaissance de plaque d'immatriculation, suivi de véhicules, détecteur d'objets.

%\textbf{\Large{Résumé}}\\
% essayer de faire le résumé en Arabe et en Anglais
%\par
%La gestion de parking intelligente transforme les approches classiques grâce à l'incorporation de technologies avancées. Son but est d'optimiser l'usage des places de stationnement pour une expérience aisée des conducteurs, en automatisant le suivi des véhicules, l'attribution d'emplacements, la gestion des paiements et la sécurité. L'objectif est de diminuer les embouteillages, faciliter le stationnement, augmenter la satisfaction des clients et maximiser l'utilisation de l'espace disponible. 
%\par
%Habituellement, la gestion de parking requiert l'enregistrement des heures d'entrée/sortie des véhicules et de leurs plaques d'immatriculation, avec un calcul précis des places disponibles. L'ensemble de ces tâches sont effectuées par un employé spécialisé qui se concentre sur l'accueil des véhicules à leur arrivée, la vérification des autorisations et l'enregistrement des informations.
%\par
%Dans ce contexte, notre travail consiste à développer un système intelligent en temps-réel  pour la gestion de parking en utilisant de l’intelligence artificielle. Ce système automatisé l’identification des véhicules, suit leurs mouvements, détecte les plaques d’immatriculation et calcule le nombre de places disponibles et vacantes dans le parking sans intervention humaine. En utilisant le modèle YOLOv8, nous avons testé cette solution avec des images de voitures et de plaques d'immatriculation Algériennes collectées de Ouedkniss et Facebook, montrant ses performances face à EasyOCR et OpenALPR.

%\textbf{Mots clés: } Gestion de parking, système en temps réel, détection de véhicules, reconnaissance de plaques d'immatriculation, intelligence artificielle.

%La bonne gestion des parkings nécessite d'enregistrer l'heure d'entrée et de sortie de la voiture avec l'enregistrement de sa plaque d'immatriculation et vous devez connaître le nombre de places disponibles et ces opérations sont généralement effectuées par un employé spécialisé avec l'aide de quelques outils et programmes
%Le rôle de l'employé est généralement lorsqu'un véhicule arrive au stationnement .
%\begin{itemize}
 %   \item Connaître le nombre de places vides.
 %   \item Enregistrement d'une plaque d'immatriculation.
  %  \item Déterminez si le véhicule est autorisé à entrer.
%    \item Notez l'heure de son entrée.
%\end{itemize}
%Lorsque la voiture quitte le parking, l'employé Calcul de la durée de stationnement.\\
%\par
%La vision par ordinateur en temps réel est un domaine de recherche en informatique qui vise à permettre aux ordinateurs de comprendre et d’interpréter les images et les vidéos en temps réel. Cependant, il y a plusieurs défis à relever pour que cela soit possible. Par exemple, la vision par ordinateur doit être capable de traiter des images en temps réel, ce qui nécessite des algorithmes de traitement d’image très rapides et efficaces. De plus, la vision par ordinateur doit être capable de traiter des images dans des conditions difficiles, telles que des environnements peu éclairés ou des images floues. Enfin, la vision par ordinateur doit être capable de traiter des images dans des contextes complexes, tels que des scènes avec plusieurs objets en mouvement.\cite{lebigdata_01} \cite{ibm_01}\\
%\par
%Dans ce mémorandum, nous souhaitons concevoir et mettre en œuvre un système capable d'effectuer les tâches des employés et de faire en sorte que le processus de stationnement se déroule de manière automatique, à l'aide de la vision par ordinateur en temps réel et de l'intelligence artificielle, en créant un apprentissage en profondeur modèles capables de reconnaître les véhicules, les plaques d'immatriculation et les mouvements des véhicules.

%\textbf{Mots clés :} gestion des parkings, temps réel, vision par ordinateur , l'intelligence artificielle , apprentissage en profondeur ,reconnaître les véhicules , reconnaître les plaques d'immatriculation  .